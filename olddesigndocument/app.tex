\chapter{Appendix}

\section{Risk Analysis}
It is important to account for any risks that may happen during the development of this architecture. By establishing a contingency plan for common risks, hindrances within the development process will be minimized. The following risks can seen in Table T-1 below.

\begin{table}
\caption{Risk Table}
\label{arm:table}
\begin{center}
\begin{tabularx}{\textwidth}{|l|l|l|l|l|l|}\hline
	Risk & Consequence & Probability (0-1) & Severity (0-10) & Impact (Prob.*Sev.) & Mitigation Strategy
	\\\hline
	Bugs & Unable to meet requirements. Program does not work properly & 0.7 & 7 & 4.9 & Test for bugs early and often. Attend to bugs as soon as possible
	\\\hline
	Lack of Technical Ability & Delays. Unable to meet requirements & 0.3 & 7 & 2.1 & Work ahead of schedule. Practice
	\\\hline
	Changing Requirements & New features are needed & 0.2 & 5 & 1.0 & Maintain code flexibility. Communicate with customer
	\\\hline
	Loss of Data & Modules will have to be rebuilt & 0.1 & 9 & 0.9 & Version control systems
\end{tabularx}
\end{center}
\end{table}

\clearpage
\newpage

\section{Development Timeline}
The development timeline shows a clear view of each team member’s responsibilities and roles in this project. Each team member’s progress can be traced across the time periods shown below in Table T-2.

%%\begin{figure}[h]
%%	\caption{Development Timeline}
%%	\includegraphics[width=8cm]{myFile}
%%	\centering
%%\end{figure}