\chapter{Introduction}
\section{Problem Statement}

Santa Clara University students spend much of their academic career completing coursework material or studying for exams. Students within the same class often meet with one another outside of the classroom, either in person or through social media. Group meetings are great for studying for a class, but students have very busy schedules, so setting up a meeting time and place can be difficult. On the other hand, social media allows for students to contact each other at their own convenience, but it is usually easier to convey information in person. 

Students often use social media to organize meetings for their class.The prevalence of smartphones at SCU promotes the usage of social networking apps such as texting, Facebook, or Twitter to contact classroom peers. Access to these social resources may be convenient, but organizing meetings through these methods is inefficient due to the recreational nature of these applications. The library at Santa Clara University currently offers an online room reservation system, but the rooms are limited in both size and number. The primary concern is that the system is based on a first come, first serve basis. If a group of students attempts to reserve a room at the library, there might not be enough rooms for access due to either the limited number of rooms or the time of day. Additionally, if an individual student were to reserve a room in advance, a group of students looking for a room within the same time frame would be denied a room. Towards the middle of the school quarter, demand for these rooms rises, and finding an available room becomes increasingly more difficult. 

Our solution is to create an application dedicated to group collaboration and library reservations.  With our application, we hope that students will be able to be more efficient in planning meetings, as well as have alternative options for collaboration. After authenticating his or her status as an SCU student, the student can create an account on the system and contact other students so group study meetings can be established. In addition to enabling the student to schedule meetings, he or she will have the option to set reminders to reserve group rooms early. As an option, the system provides a classroom forum for student communication on the course material. In case a student is unable to attend a group meeting, he or she can use this feature to remotely collaborate with his or her group members. This application will not be limited to only organizing meetings for a library room, but also for non-academic events such as off-campus gatherings or club sessions. By giving students within the same class the ability to easily form meetings and reserve rooms within the same interface,  organizing study groups will be greatly simplified for students.

\section{Project Overview}

Our project is the development of a mobile application that will serve as a platform for student collaboration. With it, students will be able to share information through the app, as well as set up meetings. We also hope that we will be able to incorporate a system that will promote usage of the library’s study rooms.

\section{Background}

When this project topic came up, we were interested because it could prove to be a valuable resource to many people. With the ongoing development in technology, we believe that one of the most useful applications of technology is information sharing. This is especially relevant to students, who need to increase their knowledge not only for classes, but also life after graduation. It is important to point out that many students do collaborate with one another. It is not uncommon for them to exchange knowledge online or set up appointments to study. Our goal with our application is to make this process more efficient for a wider variety of students.

\section{Research}

The research that we have done on knowledge sharing and collaboration will act as support for our project. The sources we found showed that collaboration is a valuable asset to improve learning. For example, the paper by Cheng demonstrated that devices can be used during lecture to promote interactivity and group thinking. While we do not plan to have our app be used in a lecture setting, we were able to see some different ways that collaborative applications can help students learn.

Another source, written by Rossitto and Eklundh, supports our idea that students should have a dedicated study area. They compare students to “nomads” since they often have to roam from place to place when completing schoolwork. In a campus environment, it can be difficult to find a suitable workplace that is consistently available for (group) use. For Santa Clara, the most reliable places are the study rooms in the library. We hope that our app will allow students to schedule group meetings in these rooms earlier.

A third source that we found to be interesting, was “Knowledge Sharing…”, which is an expanded view of our topic. Our target audience is students at SCU who may have trouble meeting groups due to various circumstances. The application mentioned in this source deals with students from various schools who lack adequate sources of knowledge, such as poor academic institutions. This issue is likely to present itself in many parts of the world, and collaboration applications are useful.