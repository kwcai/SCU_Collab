\chapter{Introduction}
\section{Problem Statement}

Collaboration is a very useful resource for learning. For example, students within the same class often try to collaborate with one another outside of the classroom, either in person or through technology. Group meetings are great for studying for a class, but students have very busy schedules, so setting up a meeting time and place can be difficult. On the other hand, technology allows for students to contact each other at their own convenience, but it is usually easier to convey information in person. 

Students often use social media to organize meetings for their class. The prevalence of smartphones at SCU promotes the usage of social networking apps such as texting, Facebook, or Twitter to contact classroom peers. For example, Facebook provides a service to create group pages with limited access. This system can be used for group discussion as well as setting up events. Access to these social resources may be convenient, but organizing meetings through these methods is largely inefficient due to the recreational nature of these applications. As another option, Camino is SCU's academic system which professors and students use. One of Camino's features are the class discussion boards, which classmates may use to interact online. While it is a useful feature, there is lack of activity possibly due to limited freedom. Students are grouped by class section, and groups can only be formed within those classes.

Our solution is to create an application dedicated to group collaboration.  With our application, we hope to provide students with the ability to efficiently schedule meetings and have focused discussion. After authenticating his or her status as an SCU student, the student can form groups with other students. Within these groups, there will be a forum, in which they can discuss class materials, and an event planner, in which they can create and view events. With these two options, students have the ability to meet in person and conveniently collaborate online. In case a student is unable to attend a group meeting, he or she can use this feature to remotely collaborate with his or her group members. By giving students within the same class the ability to easily form meetings,  organizing study groups will be greatly simplified for students.

\section{Project Overview}

Our project is the development of a mobile application that will serve as a platform for student collaboration. With it, students will be able to share information through the application as well as set up meetings.

\section{Background}

When this project topic came up, we were interested because it could prove to be a valuable resource to many people. With the ongoing development in technology, we believe that one of the most useful applications of technology is information sharing. This is especially relevant to students, who need to increase their knowledge not only for classes, but also life after graduation. It is important to point out that many students do collaborate with one another. It is not uncommon for them to exchange knowledge online or set up appointments to study. Our goal with our application is to make this process more efficient for a wider variety of students.

\section{Research}

The research that we have done on knowledge sharing and collaboration will act as support for our project. The sources we found showed that collaboration is a valuable asset to improve learning. For example, the paper by Cheng demonstrated that devices can be used during lecture to promote interactivity and group thinking~\cite{Chang2015}. While we do not plan to have our app be used in a lecture setting, we were able to see some different ways that collaborative applications can help students learn.

Another source, written by Rossitto and Eklundh, supports our idea that students should have a dedicated study area~\cite{Rossitto2007}. They compare students to “nomads” since they often have to roam from place to place when completing schoolwork. In a campus environment, it can be difficult to find a suitable workplace that is consistently available for (group) use. For Santa Clara, the most reliable places are the study rooms in the library. We hope that our app will allow students to schedule group meetings in these rooms earlier for future use.

Ogunde Opeoluwa and his team also wrote a paper that expanded our view of the topic~\cite{Opeoluwa2011}. Our target audience is students at SCU who may have trouble meeting groups due to various circumstances. The application mentioned in this source deals with students from various schools who lack adequate sources of knowledge, such as poor academic institutions. This issue is likely to present itself in many parts of the world, and collaboration applications are useful.

\bibliographystyle{plain}