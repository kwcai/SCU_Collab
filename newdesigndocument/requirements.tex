\chapter{Requirements}
The following requirements describe how the major functions of the system will behave as proposed in the introduction. Functional requirements define actions the system will explicitly perform. Non-functional requirements describe the means by which functional requirements will be achieved. Both functional and non-functional requirements will be categorized as either critical, recommended, or suggested based on the importance of a task. There will be several design constraints that will define the creation of the system.

\section{Functional Requirements}
\subsection{Critical}
	\begin{itemize}
		\item The application must allow students to create and join groups
		\item The application must provide a forum for discussion
		\item The application must display a calendar of group meeting dates
	\end{itemize}
\subsection{Recommended}
	\begin{itemize}
		\item The application will allow students the option to reserve a library room
		\item The application will show all members of a group
	\end{itemize}
\subsection{Suggested}
	\begin{itemize}
		\item The application will allow students to directly message group members
	\end{itemize}

\section{Non-functional Requirements}
\subsection{Critical}
	\begin{itemize}
		\item The application will allow concurrent users at the same time
		\item The application will protect the privacy of the users
		\item The application will be user friendly and intuitive
	\end{itemize}
\subsection{Recommended}
	\begin{itemize}
		\item The application will have a clean interface for ease of use
		\item The application can be easily updated or changed if needed
	\end{itemize}

\section{Design Constraint}
	\begin{itemize}
		\item The system must be able to run on Android mobile phones
	\end{itemize}